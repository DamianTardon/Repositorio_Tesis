\documentclass[11pt]{report} % Clase de documento para un informe
% Documento y codificación general
\usepackage[utf8]{inputenc} % Tildes, "ñ" y símbolos especiales.
% Idioma y traducción
\usepackage[spanish,es-tabla]{babel} % Traduce el documento al español.
\usepackage[a4paper,left=3cm,right=2.5cm,top=2.5cm,bottom=2.5cm]{geometry} % Permite cambiar el tamaño de la página.
% Imágenes y figuras
\usepackage{graphicx} % Permite incluir imágenes.
\usepackage{subcaption} % Permite agregar subfiguras dentro de una figura principal, cada una con su leyenda.
\usepackage{float} % Permite fijar la posición de las figuras y tablas [H].
% Matemáticas y símbolos
\usepackage{amsmath} % Permite usar entornos matemáticos avanzados.
\usepackage{amssymb} % Permite usar símbolos matemáticos adicionales.
\usepackage{mathtools} % Permite usar herramientas matemáticas avanzadas.
\usepackage{multirow}  % Para texto en varias filas
\usepackage{booktabs}
% Hipervinculos y referencias
\usepackage[hidelinks]{hyperref} % Permite crear hipervínculos transparentes dentro del documento.
\usepackage[table,xcdraw]{xcolor} % Permite usar colores textos y tablas.
\usepackage[backend=biber]{biblatex} % Permite gestionar bibliografía.
\usepackage{csquotes} % Mejora el formato de citas textuales.

\begin{document}
Página principal.\\

Como va?

\begin{table}[H]
    \centering
    \begin{tabular}{|c|c|c|}
        Marca & Modelo & Ancho de banda & Tasa de muestreo & Categoría eléctrica & Tensión máxima \\
        GW Instek & GDS-1022 & 25 MHz & 250 MSa/s & CAT II & 300 V \\
    \end{tabular}



\begin{table}[H]
    \centering
    \begin{tabular}{|c|c|} \hline 
         $R_L~[k\Omega]$& $V_{in}~[V]$\\ \hline 
         $1$& $-9  ~a~7,8$\\ \hline 
         $2$& $-4,5~a~3,9$\\ \hline 
         $5$& $-1,9~a~1,6$\\ \hline 
         $10$&$-0,9~a~0,8$\\ \hline
    \end{tabular}
    \caption{Rango de $V_{in}$ en función de $R_L$}
    \label{tab:Vin=f(RL)}
\end{table}
\end{table}
\end{document}